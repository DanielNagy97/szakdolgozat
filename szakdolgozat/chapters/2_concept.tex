\Chapter{Koncepció}

\Section{Kiterjesztett valóság}

% TODO: A dolgozat témája hogyan kapcsolódik hozzá, és hogy általában mit értünk alatta.

A kiterjesztett valóság (\textit{Augmented Reality}, AR), mint fogalom alatt olyan érezékelhető valóságot értünk, amelyben a valós és virtuális elemek keverednek és a virtuális elemekre a valódi világ hatással van. A virtuális valóssággal (\textit{Virtual Reality}, VR) ellentétben, ahol az érzékelt valóság teljes mértékben digitális, az AR-nek a környezetét a valós világ adja, azzal szoros kapcsolatban áll és, mint a nevében is benne van, annak egyfajta kiterjesztésével kíván egy új valóságot létrehozni.

Megvalósításának számos módja ismert, gondolhatunk például a legelterjettebb felhasználási területre, az okostelefonos megoldásokra is. A legtöbb közösségi felülethez tartozó kamera-alkalmazások kiterjesztett valóság alapú technológiákat is alkalmaznak, melyek segítségével az előlapi kamera képével a felhasználók az arcukra különböző filtereket, digitális maszkokat illeszthetnek. Megemlíteném még a népszerű játékot is, a \textit{Pokemon Go}-t, melyben a képernyőt nézve nyerhetünk betekintést a kiterjesztett valóságba. Az okostelefon különféle szenzorainak hála (giroszkóp, GPS, iránytű) a kiterjesztett világot bejárhatjuk, a telefon képernyője egy ablakot nyit eme kevert valóságba. Az ilyen megoldásokat \textit{Spatial Augmented Reality}-nek \cite{bimber2005spatial}, vagyis térbeli kiterjesztett valóságnak nevezzük. A virtuális elemek helyzete a valós világ bizonyos pontjaihoz van kötve, a virtuális elemek függenek a valós környezettől.

A kevert világ észlelésének módjai lehetnek a már említett kézi eszközökön kívül fejre helyezhető eszközök is, illetve megvalósulhat külső eszközök használatával is, amelyek nem sokkal térnek el az előbbiektől.
A fejre helyezhető eszközök általában olyan szemüvegek, amelyekben a valós és virtuális világ egyformán érzékelhető. Ilyen eszköz például a \textit{Microsoft HoloLens} is. Ezen eszközökre is jellemző a már okostelefonoknál említett szenzorok használata a pontosabb eredmény elérése érdekében.

A külső eszközökkel megvalósított kiterjesztett valóság, amelyet több személy is egyaránt megfigyelhet, a különböző projektorok, hologrammok és egyéb speciális technikák alkalmazásával valósulhat meg. Illetve a már okostelefonoknál említett megoldás felnagyított valtozata is használható, fix kamerával és nagyobb képernyővel. Ha pedig a kamera képét függőlegesen tükrözzük, akkor olyan hatást érhetünk el, mintha egy tükrön keresztül vizsgálhatjuk a körülöttünk létrehozott kevert valóságot. A tükör képe a videófolyam, amelyet virtuális elemekkel egészítünk ki, a telefonos megoldáshoz hasonlóan. A kiterjesztett valóság ilyen formája megfelelő méretű képernyőt használva felhasználható lehetne például múzeumokban, vidámparkokban és egyéb közösségi helyszíneken is. A dolgozatomhoz elkészített programom vezérlése szempontjából ezen megoldása tűnt a legmegfelelőbbnek. A prezentáló személy a tükörképe segítségével kapcsolatba tud lépni a kiterjesztett valósággal és annak elemeivel, így irányítva a program működését.

\Section{Prezentációs szoftverek}

% TODO: Megnézni 3-6 hasonló szoftvert. Készülhet hozzá összehasonlító táblázat, képernyőképek.

Manapság számtalan prezentációs szoftver közül választhatunk. A legtöbb szoftverről elmondható, hogy diasoros elven működik, vagyis a prezentáció anyaga egy előre meghatározott sorrendben fut le, a tartalom pedig ún. fóliákon helyezkednek el, hasonló módon, mint a régi írásvetítők esetében, ahol a prezentáció témáját előre elkészített fóliák segítségével mutathattuk be. Manapság már a prezentáció anyaga teljes mértékben digitális, viszont a hagyományos értelemben vett prezentációs szoftverek esetében a vetítés menete hasonló, a digitális fóliákat cserélgethetjük, lépkedhetünk közöttük.

A digitális tartalom lehetőséget ad arra, hogy dinamikus elemekkel ellátott, látványos prezentációt készítsünk. Erre a különböző szoftverek különféle lehetőségeket kínálnak.
Az egyik legismertebb prezentációs szoftver kétségkívül a \textit{Microsoft Power Point}. Meghatározó szereplője a prezentációs szoftverek világának. Használatával gyorsan és egyszerűen állíthatunk össze prezentációkat, amelyek minőségét a látványos átvezető animációk és a dinamikus elemek használatával fokozhatunk. Hátrányai közé tartozik a platformfüggőség és a magas ára is.

Több, hasoló elven működő ingyenes szoftver is megjelent az évek során, amelyek képesek kezelni a PPT formátumát is. Ezen szoftverek segítségével megoldhatjuk a kompatibilitási gondokat. Mivel ingyenesek és platformfüggetlenek, mindenki számára elérhetőek. Megemlíthetném a sok közül a \textit{Libre-Office}-t, a \textit{WPS-Office}-t és a \textit{Google Slides}-ot. Működésük nagyban hasonlít a \textit{Power Point}-hoz, utóbbit internetkapcsolat mellett a böngénszőnkben használhatjuk.

A diasoros prezentációs szoftverek világán túl is léteznek izgalmas megoldások. 
Ilyen a \textit{Prezi} szoftvere is, mely újítása egy egyszű koncepció nagyszerű megvalósítása: A tartalmat egy nagymértű virtuális vászonra helyezhetjük el, majd megadhatjuk, hogy a nézőpont mely pontok között ugráljon. Így olyan hatást érhetük el, mintha a lebegnénk a virtuális vászon fölött. Ha szeretnénk lehetőségünk van megmutatni a teljes vásznat is. A \textit{Prezi} a korábbi megoldások lineáris kötöttségét kívánta eltörölni. Használatával egyedi és látványos bemutatókat készíthetünk. Az ingyenes változata internet használata mellett böngészőben használható.
2017-ben jelent meg a hír, hogy a Prezi kiterjetsztett valóságot megvalósító prezentációs szoftvert tervez fejleszteni, \textit{Prezi AR} néven. A beharangozó konferencián kívül az internetre felkerült még pár videó is, amelyekkel a koncepciót kívánták szemléltetni. A tervek szerint a prezentáló a videófolyamra rajzolt virtuális elemek segítségével mutathatja be a témakört. A közzétett demo videók alapján kapcsolatba is léphet ezekkel az elemekkel a gesztusai segítségével. A projektről azóta semmiféle hír nem jelent meg.

A kiterjesztett valóság adta lehetőségek prezentációs szoftverek használatánál szemléltetésére jó példa a már elkészült szoftver, amely a \textit{Interactive Body-Driven Graphics for Augmented Video Performance} című publikációban jelent meg, 2019-ben. \cite{saquib2019interactive}
A megvalósításhoz a \textit{Kinect} kamerarendszert használták fel, amely mélységérzékelővel is fel van szerelve.
Kezdetben \textit{Deep-Learning} technológián alapuló alakferismerést kívántak alkalmazni, viszont ez a megoldás még egy magas teljesítményű számítógépen sem tudott kielégítő futási eredményeket adni. A valós idejű futás, amely egy prezentációs szoftvernél elvárt tulajdonság, nem volt lehetséges ezzel a technikával. A szoftver, amelyet használni szerettek volna az \textit{OpenPose} volt, amely magas pontossággal képes megbecsülni a 2D-s videófolyamon elhelyezkedő személyek helyzetét és azok testrészeinek a pozícióját. \cite{cao2018openpose}
A cikkben leírtak szerint a \textit{Kinect} használata megkönnyítette a fejlesztést és segítségével képes valós időben futni a szoftver.
Egy \textit{Widget} szerkesztőt is készítettek a programhoz, amellyel az egérrel vagy digitális rajztáblán rajzolt virtuális elemeket szerkeszthetünk. Az elemekkel a prezentáció során az előadó kapcsolatba tud lépni. A szoftver használata során elengedhetetlen a léptető használata. Ezzel a prezentáló személy egyrészt változtathatja a megjelenített elemeket, aktiválhatja a funkciókat. Másrészt a magyarázó mozgás is jelezhető vele, ilyenkor nem kerülhet interakcióba az elemekkel az előadó.

% A megvalósított koncepció eszközfüggőségének hátránya (kinect, léptető)

% Esetleg összehasonlítás a sajátommal, ha már működik

\Section{Valós idejű videó annotálás}

% TODO: OpenCV-ről néhány dolgot megemlíteni.

Az \textit{OpenCV} egy \textit{open source}, elsősorban valós idejű számítógépes látás megvalósítására támogatást nyújtó, magas tudású függvénykönyvtár. Számos képfeldolgozó algoritmus található benne, melyek implementációjánál a fejlesztők a lehető legjobb teljesítmény elérésére törekedtek.

Az \textit{OpenCV} eredetileg C és C++ nyelven íródott, de különböző nyelveket is támogat, mint például a \textit{Python}-t, \textit{Ruby}-t, \textit{Matlab}-ot, stb\ldots
Platformfüggetlen, így \textit{Linux}-on, \textit{Windows}-on és \textit{Mac OS X} rendszereken is használható. Lehetőséget nyújt továbbá a párhuzamosításra is a \textit{CUDA} és az \textit{OpenCL} technológiák segítségével és tartalmaz egy általános célú gépi tanulásos könyvtárat is, melyek tovább szélesítik a felhasználási területeinek körét. \cite{bradski2008learning}

Videófolyamok feldolgozásához is hasonló eljárásokat alkalmazhatunk, mint az önálló képek esetében. A videó képkockáit iteratív módon dolgozhatjuk fel, képkockáról képkockára. A videó két forrásból származhat: fájból és külső eszközből (pl.: webkamera képe). Az \textit{OpenCV}-ben mindkét esetben először definiálnunk kell egy olvasó eszközt. Ezt \textit{Python} esetében a \texttt{cv2.VideoCapture()} metódussal érhetjük el. Fájlból történő olvasás esetén paraméterként a fájl elérési útvonalára kell hivatkozni, külső eszköz esetén pedig az eszköz \textit{index}-ével történik a hivatkozás. Ha egyetlen ilyen eszköz áll rendelkezésünkre, akkor a 0-ás indexet is használhatjuk, hiszen az az első elérhető eszközre mutató index.
Az eszköz definiálása után a \texttt{.read()} metódusával olvashatjuk ki a képkockákat. A metódus meghívásakor visszatérési értékként megkapjuk a soron következő képkockát és egy értéket, amely az olvasás sikerességéről tájékoztat bennünket. A metódus meghívásakor a videóolvasó pointerét is inkrementáljuk, így a következő \texttt{.read()} meghívásakor már a soron következő képkockát olvashatjuk ki a videófolyamból. Ha már nem kívánjuk tovább használni az adott erőforrást, akkor a \texttt{.release()} metódusával szabadíthatjuk fel azt.

\Section{Gépi tanulás}

SciKit Learning
