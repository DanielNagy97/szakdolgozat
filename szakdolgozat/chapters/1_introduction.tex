\Chapter{Bevezetés}

% TODO: Leírni, hogy ez miben lesz újabb, jobb mint az eddigiek?

% A fejezet célja, hogy a feladatkiírásnál kicsit részletesebben bemutassa, hogy miről fog szólni a dolgozat.
% Érdemes azt részletezni benne, hogy milyen aktuális, érdekes és nehéz probléma megoldására vállalkozik a dolgozat.

% Ez egy egy-két oldalas leírás.
% Nem kellenek bele külön szakaszok (section-ök).
% Az irodalmi háttérbe, a probléma részleteibe csak a következő fejezetben kell belemenni.
% Itt az olvasó kedvét kell meghozni a dolgozat többi részéhez.

% Ezek még csak ötletek/egy gyors vázlat

A kiterjesztett valóság adta lehetőségek feltérképezése egy izgalmas kutatási terület. Szorosan összefonódik a képfeldolgozás, az ember-gép interakció (HCI) témakörökkel és ezáltal a gépi tanulásos témakörrel is. Vagyis kutatási területek széles skáláját érinti, amelyekkel manapság kiemelkedően sok cikk foglalkozik. Egy rendkívül izgalmas témakör, rengeteg lehetőséggel.

A dolgozatom feladataként egy olyan megoldás után kutatok a fent említett témakörökön belül, mely segítségével össze lehetne állítani egy elsősorban prezentációs szoftverként használatos alkalmazást, kihasználva a témakörökben rejlő lehetőségeket, a technikai korlátainkat figyelembe véve. A programom használatával a prezentáló személy a videófolyamra illesztett virtuállis elemek segítségével mutathatja be a témakört, a gesztusai segítségével vezérelheti a prezentáció menetét. A programom, felhasználási területét tekintve elsősorban élő közvetítések, konferenciahívások alkalmával lehetne hasznos. Így tehát a hagyományos értelemben vett prezentációs formákat nem célszerű helyettesíteni vele. Viszont az online felületen történő alkalamazása számos előnnyel járhat. Segítségével látványos, újszer előadásokat lehetne tartani.

A koncepciót igyekeztem a technikai lehetőségekhez mérten összeállítani. A későbbiekben megemlített korábbi kutatásokból láthatjuk, hogy előszeretettel használnak különféle segédeszközöket, mint például a mélységérzékelővel is ellátott \textit{Kinect} kamerarendszert. Az ilyen megoldások hátránya egyértelműen az eszközfüggőség, hiszen ezen szoftverek használatához elengedhetetlen ezen eszközök megléte. Az ilyen eszközökből eredő hibákkal is kénytelenek vagyunk együtt élni, mint például a \textit{Kinect}-re jellemző alacsony felbontás. Az elkészített programomhoz csupán egy kameraeszközre van szükségünk, ami akár lehet egy laptop egyszerű webkamerája is.

A dolgozat elején rövid ismertetők formájában bemutatásra kerülnek a témához kapcsolódó fogalmak, mint a kiterjesztett valóság, a prezentációs szoftverek, a képfeldolgozás és a gépi tanulás. Majd a definiált gesztusokról és lehetséges megvalósítási módjairól, a program működési módjáról olvashatunk áttekintő leírást. Ezt követi a programomban található virtuális elemek (widgetek) megvalósítási módjainak részletezése.

A megvalósítás fejezetben az elkészített implementációról olvashatunk pontos leírást, bemutatva az \texttt{arpt} csomag felépítését és a program szerkezetét.

A dolgozatot végül két darab példa prezentációval kívánom zárni, melyeken keresztül bemutatom az elkészített programom lehetséges felhasználási módját, működését.
