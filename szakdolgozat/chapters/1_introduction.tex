\Chapter{Bevezetés}

% TODO: Leírni, hogy ez miben lesz újabb, jobb mint az eddigiek?

% A fejezet célja, hogy a feladatkiírásnál kicsit részletesebben bemutassa, hogy miről fog szólni a dolgozat.
% Érdemes azt részletezni benne, hogy milyen aktuális, érdekes és nehéz probléma megoldására vállalkozik a dolgozat.

% Ez egy egy-két oldalas leírás.
% Nem kellenek bele külön szakaszok (section-ök).
% Az irodalmi háttérbe, a probléma részleteibe csak a következő fejezetben kell belemenni.
% Itt az olvasó kedvét kell meghozni a dolgozat többi részéhez.

% Ezek még csak ötletek/egy gyors vázlat

A kiterjesztet valóság adta lehetőségek feltérképezése egy izgalmas kutatási terület. Szorosan összefonódik a képfeldolgozás, az ember-gép interakció (HCI) témakörökkel és ezáltal a gépi tanulásos témakörrel is. Vagyis kutatási területek széles skáláját érinti, amelyekkel manapság kiemelkedően sok cikk foglalkozik \ldots

A dolgozatom feladataként egy olyan megoldás után kutatok a fent említett témakörökön belül, mely segítségével össze lehetne állítani egy elsősorban prezentációs szoftverként használatos alkalmazást, kihasználva a témakörökben rejlő lehetőségeket, a technikai korlátainkat figyelembe véve. A programom használatával a prezentáló személy a videófolyamra illesztett virtuállis elemek segítségével mutathatja be a témakört, a gesztusai segítségével vezérelheti a prezentáció menetét. A dolgozatban megemlített, már megjelent kutatási eredmények alapján láthatjuk, hogy az ilyen jellegű prezentációs szoftverek alkalmazása mennyire újszerű terület \ldots

A dolgozatomban igyekeztem a technikai lehetőségekhez mérten összeállítani a koncepciót. A korábbi kutatásokban előszeretettel használnak különféle segédeszközöket, mint például a mélységérzékelővel is ellátott \textit{Kinect} kamerarendszert. Az ilyen megoldások hátránya egyértelműen az eszközfüggőség, hiszen ezen szoftverek használatához elengedhetetlen ezen eszközök megléte. Az ilyen eszközökből eredő hibákkal is kénytelenek vagyunk együtt élni, mint például az alacsony felbontás. A dolgozatomhoz elkészített programomhoz csupán egy kameraeszközre van szükségünk, ami lehet egy laptop egyszerű webkamerája is.

Az elkészítendő programom felhasználási területét tekintve hasznos lehet online konferencia előadások vagy élő közvetítések alkalmával is \ldots

A dolgozat elején rövid ismertetők formájában bemutatásra kerülnek a témához kapcsolódó fogalmak, mint a Kiterjesztett valóság, a prezentációs szoftverek, a képfeldolgozás és a gépi tanulás. Majd a definiált gesztusokról és lehetséges megvalósítási módjairól, a program működési módjáról olvashatunk áttekintő leírást. Ezt követi a programomban található virtuális elemek (widgetek) megvalósítási módjainak részletezése.

A megvalósítás fejezetben az elkészített implementációról olvashatunk pontos leírást, bemutatva az \texttt{arpt} csomag felépítését és a program szerkezetét.

A dolgozatot végül két darab példa prezentációval kívánom zárni, melyeken keresztül bemutatom az elkészített programom lehetséges felhasználási módját, működését.
