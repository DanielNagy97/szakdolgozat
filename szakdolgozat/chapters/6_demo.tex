\Chapter{Prezentáció példák}

% TODO: Tutorial jelleggel leírni, és bemutatni, hogy hogy lehet használni az alkalmazást.

A program jelenlegi verziójához nincs mellékelve grafikus felülettel ellátott projekt-szerkesztő, viszont a projekt sturktúraszabályait követve létrehozhatunk új prezentáció leíró állományokat.

\bigskip

\dirtree{%
.1 /presentation\_project01.
.2 /images.
.3 button.png.
.3 tuner.png.
.3 \ldots.
.2 /preferences.
.3 canvasses.json.
.3 settings.json.
.3 windows.json.
.2 scene.json.
}

\bigskip

\noindent A projekt fájl \textit{JSON} állományokból, képekből és egy \texttt{actions.py} modulból áll, amelyben leírásokat adhatunk a \textit{Button} és a \textit{Tuner} widgetek viselkedésére.

A \texttt{/preferences} jegyzékben pedig általános beállításokra vonatkozó leírásokat adhatunk meg. A \texttt{settings.json}-ben a \textit{videófolyam}, a \textit{Rács} és a \textit{Hőtérkép} paramétereit adhatjuk meg.
\texttt{canvasses.json}-ben pedig, ha igényt taratunk rá, a \textit{Vektormező} és a \textit{Globális eredővektor grafikon} vásznak tulajdoságait állíthatjuk be.
\texttt{windows.json}-ben pedig az OpenCV által megjelenítendő ablakokat állíthatjuk be. A prezentációs során csak egy teljesképernyős ablakra van szükségünk, melyben a kimeneti képet láthatjuk. Ha csak az eredmény érdekel bennünket és nem tartunk igényt a program működésének megfigyelésére, akkor az utóbbi két fájl üresen hagyható.

A \textit{DataParser} egység feldolgozza a felhasználó projekt fájlait és a bennük található leírások szerint futtatja a programot.

A \texttt{scene.json} állományban definiálhatjuk az egyes diák tartalmát. Az állomány egy lista, melynek elemei a \textit{JSON} objektummal leírt diák. A jelenlegi verzióban csupán egyetlen mezőt tartalmaznak a diákat leíró objektumok, a \texttt{widgets}-et. Ha később további információkkal szeretnénk ellátni a diákat (például metadatokat, egyéb leírásokat, további funkciókat szeretnénk jelölni), ezen a szinten tehetjük meg. A \texttt{widgets}-en belül pedig a widgetek listáját találjuk. Az egyes widget típusok különféle paraméterekkel rendelkeznek.\\
Példaként tekintsünk egy egyetlen diával rendelkező \texttt{scene.json} állományt, melyen egyetlen \textit{Tuner} widget helyezkedik el:
\begin{verbatim}
[
    {
        "widgets": [
            {
                "type":      "Tuner",
                "position":  [420, 300],
                "dimension": [150, 150],
                "image":     "images/tuner.png",
                "min_value": 80,
                "max_value": 200,
                "action":    "sample_action"
            }
        ]
    }
]
\end{verbatim}
Jól látható, hogy a widgeteket \textit{JSON} objektumokként írhatjuk le, \texttt{type} mezőben jelöljük a widget típusát, az utána következő mezőkben pedig a widget paramétereinek értékeit adhatjuk meg. Az \texttt{action} mezőben pedig a widget-hez tartozó függvény nevét jelölhetjük, az \texttt{actions.py}-ból.

A következőkben két példa prezentáción keresztül szeretném bemutatni a program használatát, az elkészült állapotok lehetőségeit kihasználva. Az első téma \ldots

\Section{Téma 1}


\Section{Téma 2}
