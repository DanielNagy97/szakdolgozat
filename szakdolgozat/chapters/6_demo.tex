\Chapter{Prezentáció példák}

% TODO: Tutorial jelleggel leírni, és bemutatni, hogy hogy lehet használni az alkalmazást.

A program felhasználási területe elsősorban 

A program jelenlegi verziójához nincs mellékelve grafikus felülettel ellátott projekt-szerkesztő, viszont a projekt sturktúraszabályait követve létrehozhatunk új prezentáció leíró állományokat.

\bigskip

\dirtree{%
.1 /presentation\_project01.
.2 /images.
.3 button.png.
.3 tuner.png.
.3 \ldots.
.2 /preferences.
.3 canvasses.json.
.3 settings.json.
.3 windows.json.
.2 scene.json.
}

\bigskip

\noindent A projekt fájl \textit{JSON} állományokból, képekből és egy \texttt{actions.py} modulból áll, amelyben leírásokat adhatunk a \textit{Button} és a \textit{Tuner} widgetek viselkedésére.

A \texttt{/preferences} jegyzékben pedig általános beállításokra vonatkozó leírásokat adhatunk meg. A \texttt{settings.json}-ben a \textit{videófolyam}, a \textit{Rács} és a \textit{Hőtérkép} paramétereit adhatjuk meg.
\texttt{canvasses.json}-ben pedig, ha igényt taratunk rá, a \textit{Vektormező} és a \textit{Globális eredővektor grafikon} vásznak tulajdoságait állíthatjuk be.
\texttt{windows.json}-ben pedig az OpenCV által megjelenítendő ablakokat állíthatjuk be. A prezentációs során csak egy teljesképernyős ablakra van szükségünk, melyben a kimeneti képet láthatjuk. Ha csak az eredmény érdekel bennünket és nem tartunk igényt a program működésének megfigyelésére, akkor az utóbbi két fájl üresen hagyható.

A \textit{DataParser} egység feldolgozza a felhasználó projekt fájlait és a bennük található leírások szerint futtatja a programot.

A \texttt{scene.json} állományban definiálhatjuk az egyes diák tartalmát. Az állomány egy lista, melynek elemei a \textit{JSON} objektummal leírt diák. A jelenlegi verzióban csupán egyetlen mezőt tartalmaznak a diákat leíró objektumok, a \texttt{widgets}-et. Ha később további információkkal szeretnénk ellátni a diákat (például metadatokat, egyéb leírásokat, további funkciókat szeretnénk jelölni), ezen a szinten tehetjük meg. A \texttt{widgets}-en belül pedig a widgetek listáját találjuk. Az egyes widget típusok különféle paraméterekkel rendelkeznek.\\
Példaként tekintsünk egy egyetlen diával rendelkező \texttt{scene.json} állományt, melyen egyetlen \textit{Tuner} widget helyezkedik el:
\begin{verbatim}
[
    {
        "widgets": [
            {
                "type":        "Tuner",
                "position":    [420, 300],
                "dimension":   [150, 150],
                "image":       "images/tuner.png",
                "min_value":   80,
                "max_value":   200,
                "transparent": true,
                "action":      "sample_action"
            }
        ]
    }
]
\end{verbatim}
Jól látható, hogy a widgeteket \textit{JSON} objektumokként írhatjuk le, \texttt{type} mezőben jelöljük a widget típusát, az utána következő mezőkben pedig a widget paramétereinek értékeit adhatjuk meg. A \texttt{transparent} mezőben jelölhetjük, hogy kívánjuk-e használni a widget képének $\alpha$ csatornáját. Ha \texttt{true} értékkel látjuk el ezt a mezőt, akkor a program $\alpha$ csatornával eggyütt olvasssa be a képet, \texttt{false} esetén pedig csupán csak a három színcsatornára számít. Az $\alpha$ csatornával ellátott widget képek az $\alpha$ értékeiknek megfelelően rajzolódnak ki.
Az \texttt{action} mezőben pedig a widget-hez tartozó függvény nevét jelölhetjük, az \texttt{actions.py}-ból.

A következőkben két példa prezentáción keresztül szeretném bemutatni a program használatát, az elkészült állapotok lehetőségeit kihasználva.

\Section{Képfeldolgozás, OpenCV bemutatása}

Az első példaprezentáció a képfeldolgozás témakör köré épül és bemutatásra kerül az \textit{OpenCV} is. Összeállításánal igyekeztem kihasználni a programom edddig elkészült állapotainak lehetőségeit. A videófolyamon megjelenő widget-ek képeit \textit{PhotoShop CS5} képszerkesztő segítségével szerkesztettem meg. (Ez is egy további fejlesztési irány lehet, hogy saját widget szerkesztőt is lehetne csinálni hozzá). A widgetek paramétereit a fejezet bevezetésében leírtak alapján állítottam be a \texttt{scene.json} állományban. Az állomány nyolc darab objektuma az nyolc darab diát jelöli.

Három fő részre különíthetjük el a diasor szerkezetét:
\begin{itemize}
	\item Bevezetés
	\item Menü
	\item Befejezés
\end{itemize}
Ezen szerkezeti elemek az első három diát jelölik.

Az első dián szereplő \textit{Rollable} widget tartalmazza a bevezetést, amely egy rövid áttekintés, hogy miről is fog szólni a prezentáció. Egy \textit{Shiftable} típusú \textit{OpenCV} logó is helyet kapott itt, illetve egy \textit{Button} típusú widget, melynek a funkciója a léptetés, melynek függvénye az \texttt{actions.py} modulon belül található meg.

A második diát egy menü rendszernek szántam. A csempeszerűen elhelyezkedő widgetek segítségével érhetjuk el rajtuk jelölt funkciókat. A diavetítés így nem lineáris lesz, hiszen szabad sorrendben lefuttathatjuk azokat. Ezen widgetek \textit{Button} típusúak, a hozzájuk tartozó művelet pedig az aktuális diát jelző pointer értékének változtatása a cél helyre. Ezen diák szintén a \texttt{scene.json} elemei, viszont a logiai utolsó, vagyis a sorrendben harmadik dia után helyezkednek el, tehát elérhetetlenek az utolsó diából.\\
Öt darab dia nyitható meg a csempék segítségével:
\begin{itemize}
	\item Digitális képekről
	\item Alfa csatorna, \textit{Alpha Blending}
	\item Küszöbölés technika
	\item Éldetektálás
	\item Különböző filterek/szűrők bemutatása
	\begin{itemize}
		\item Szűrő 1
		\item Szűrő 2
	\end{itemize}
\end{itemize}
Az menüből elérhető diák tartalma változó, a hozzájuk tartozó témakör kifejtését tartalmazzák. A dinamikus elemekkel ellátott diákon a témakörnek megfelelően elhelyzetem olyan vezérlőket, melyekkel manipulálhatjuk a kimeneti videófolyamot a témakörök szemléletesebb bemutatása érdekében.
Például a Küszöbölés technikát bemutató dián a funkciót aktiváló \textit{Button} megnyomása esetén a videófolyamon a \textit{Tuner} értékeinek megfelelően végrehajtódik a küszöbölés művelete. A \textit{Tuner} segítségével állíthatjuk a küszöb alsó határát, az eljárás szemléltetése érdekében.
Az éldetektálással és \textit{Alpha Blending} technikákkal foglalkozó fóliákon szintén hasonlóan demonstrálhatjuk a technikák működését, \textit{Tuner} widget segítségével értékeket állítva.
A filtereket bemutató fólia szintén egy menürendszer, melyen két \textit{Button} segítségével érhetejük el a kívánt fóliákat. Az innen elért fóliákból egy \textit{Button} segítségével léphetünk egy szintet vissza.
A módszereket bemutató diák mindegyikén elhelyezkedik egy-egy \textit{Button} is, amellyel a Menübe léphetünk vissza.

A menüből pedig szintén egy \textit{Button} segítségével léphetünk tovább a logikailag utolsó diára, ahol egy \textit{Expandable} widget segítségével foglalhatjuk össze a prezentációnkat és köszönhetjük meg a figyelmet.

\Section{Dolgozat bemutatása}

A második példaprezentáció segítségével a dolgozatomat szeretném bemutatni.
A prezentáció során a videófolyam az általános widgetek mellett valós időben frissülő ábrákkal is kiegészül, melyeket egyébként a program fejlesztői módú futása során láthatnánk. Ilyenek a \textit{Frame Difference}, \textit{Grab}, \textit{Vektormező}, stb\ldots ábrák. A folyton frissülő ábrák segítségével a dolgozat kulcs elemeit szemléletesebben bemutathatjuk.

A prezentációban a dolgozatomban tárgyalt főbb elemek kerülnek említésre. A fejezetek szerint haladva kerül bemutatásra a témakör és a dolgozat anyaga.

Az első példaprezentációban használt menürendszeres megoldást ennél is szeretném alkalmazni a prezentációk megszokott linearitását megtörve. Természetesen a prezentáció ajánlott menete így is lineáris marad, ez a fajta öszzeállítása az előadásnak csupán azért szükséges, hogy a logikailag egybefüggő részek még jobban elkülönüljenek egymástól. A fő fóliákból további alfóliák érhetők majd el. Nem célom labirintusszerűen felépíteni az előadást, hogy a prezentáló személy se tudjon kiigazodni a saját előadásában, így a lehetséges mélységi szintet minimalizálni fogom.

A prezentáció fő szerkezeti elemei a következők:
\begin{itemize}
	\item Bevezetés
	\item Koncepció
	\item Gesztusok
	\item Példa prezentációk
	\item Összegzés, befejezés
\end{itemize}
A szerkezeti elemek követik a dolgozat felépítését\ldots

Bevezetés

Koncepció
	AR, Prezik, OpenCV, Sklearn részletezése

Gesztusok
	Objektumkövetés
	rács
		ezen belül az ehhez tartozó funkciók
	hőtérkép
		az ehhez kapcsolódó megoldások
	gépi tanulásos módszerek
	
Példa prezentációk
	Opencv-s
	Dolgozatos-s - itt vissza ugrás a bevezetésre, ahonnan a következő diában a főmenübe kerülünk, szóval nem kell végigkattingatni

Összefoglalás
