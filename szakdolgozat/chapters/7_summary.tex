\Chapter{Összefoglalás}

A dolgozatban egy új, saját kezdeményezésű prezentációs szoftver elkészítése és működése került bemutatásra.
Ehhez megelőzően áttekintésre kerültek az AR/VR, a képfeldolgozás és a gépi tanulás témaköreinek alapfogalmai.
Ezt követően a prezentáció során értelmezett (a szoftver által felismerendő) gesztusok áttekintésére került sor.

A gesztusok felismerése egy aktuálisan kutatott terület. A dolgozatban bemutatott problémákhoz kapcsolódóan az eredményeket folyóiratban is sikerült publikálni \cite{nagy2020gesture}.

Az alkalmazás egy \texttt{arpt} nevű Python függvénykönyvtár formájában került megvalósításra. Ez függőségként az OpenCV és a SciKit Learn függvénykönyvtárak meglétét feltételezi. Mivel ezek több platformon elérhetők, így az elkészült alkalmazás is átvihető.

A dolgozat egy külön fejezetben két demo prezentáción mutatja be a prezentáció készítésének módját, és az elkészült prezentáció működését.

Az elkészített program további fejlesztésére több lehetséges irány is adódik.
A bemutatott gesztus detektálási módszerekben gyakran intuitív, tapasztalati úton beállított paraméterek adtak az elvárásoknak megfelelő megoldást.
A későbbiekben a hasonló képfeldolgozási, objektumfelismerési- és követési, illetve gépi tanulási módszerek tanulmányozásával megbízhatóbbá tehető az alkalmazás.

A prezentációs projektek kezeléséhez egy formátum definiálásra került, viszont érdemes azt egy grafikus szerkesztőfelülettel egyszerűbben használhatóvá, a technikai dolgokban kevésbé járatos felhasználók számára is elérhetővé tenni.

A példákban a megjelenített \textit{widget}-ek direkten az adott prezentációhoz készültek. Érdemes lehet majd előre definiált készleteket készíteni és közreadni ezekhez.
